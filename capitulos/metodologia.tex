\section{Metodologia}\label{lmetodologia}

No desenvolvimento deste trabalho será utilizado a \emph{priori}, a pesquisa 
exploratória referente a todos os métodos mais utilizados para a 
resolução de EDOs e sua aplicação em alguma linguagem de programação. 
Do ponto de vista de \citeauthoronline{prestes}: 
\begin{citacao}  
A pesquisa exploratória configura-se como a que acontece na fase 
preliminar, antes do planejamento formal do trabalho. Ela tem como 
objetivo proporcionar maiores informações sobre o assunto que vai ser 
investigado, facilitar a delimitação do tema a ser pesquisado, orientar 
a fixação dos objetivos e a formulação das hipóteses ou descobrir uma 
nova possibilidade de enfoque para o assunto (\citeyear{prestes}, p. 26).
\end{citacao}
Para o levantamento de dados e para fins de pesquisa e testes, será utilizado 
a pesquisa bibliográfica, utilizando-se principalmente de livros e
artigos sobre o assunto. De acordo com \citeauthoronline{metring} a pesquisa
bibliográfica:
\begin{citacao}
$[$...$]$ tem a finalidade de conhecer as diferentes formas de 
contribuição científica já realizadas sobre determinado assunto, 
visando encontrar dados atuais e relevantes sobre o tema investigado. 
Utiliza-se exclusivamente de material já elaborado e disponível, em 
particular livros e artigos científicos, e é a base para qualquer tipo 
de pesquisa, e também é mais ampla que a pesquisa documental. Como a 
maioria dos dados já está tratado e analisado, costuma-se nomear este 
tipo de pesquisa de secundária (\citeyear{metring}, p. 63).
\end{citacao}
