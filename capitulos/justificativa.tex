\section{Justificativa}\label{ljustificativa}

Equações Diferencias Ordinárias são ferramentas importantíssimas 
para a modelagem matemática de problemas do mundo real. Elas podem
descrever fenômenos físicos, biológicos, químicos, epidemiológicos etc.,
quando estes fenômenos são descritos em termos de taxa de variação \cite[p.~293]{filho2007algoritmos}.
Exemplos práticos que usam EDOs são, taxa de crescimento de uma
população de bactérias, taxa de crescimento de uma população de uma
cidade ou a propagação de uma doença em um grupo populacional.

Nem sempre é possível resolver uma EDO por meio analítico, ou seja,
encontrar a solução para a equação incógnita por meio de funções
elementares da matemática. Mesmo equações diferencias com aspecto 
simples como:
\begin{equation*}
y' = x^2 + y^2
%\label{eq01}
\end{equation*}
ou
\begin{equation*}
y'' = 6y^2 + x
%\label{eq02}
\end{equation*} 
não podem ser resolvidas em termos de funções elementares \cite[p.~275]{calculo}.
Um outro exemplo é a equação
\begin{equation*}
y' = 1 - 2xy
%\label{eq03}
\end{equation*}
cuja solução 
\begin{equation*}
y(x) = e^{-x^{2}} \int_0^x\! e^{t^{2}} \mathrm{d}t
%\label{eq04}
\end{equation*}
não pode ser expressa em termos de funções elementares.
%É importante destacar que as três equações citadas acima são ordinárias
%ou seja, existe apenas uma variável independente que no caso acima 
%é o $x$. Equações com mais de uma variável independente serão discutidas
%logo abaixo. 
\citeauthoronline{sperandio2003calculo} destaca,
%\citeonline[p.~231]{sperandio2003calculo} destaca 
\begin{citacao}
$[$ \ldots $]$ que em muitos casos os coeficientes ou as funções existentes  na equação diferencial são
dados somente na forma de um conjunto tabelado de dados experimentais,
o que torna impossível o uso de um procedimento analítico para determinar
a solução da equação (\citeyear{sperandio2003calculo}, p. 231).
\end{citacao}

Para as soluções numéricas não existem limitações, pois são usados valores
da variável independente e, pelo  menos, um resultado já conhecido da função.
Como esses métodos requerem cálculos em vários pontos, a resolução
manual não é nem um pouco atrativa, levando ao cansaço e muito suscetível 
ao erro. Bem, na era digital em que vivemos existe computadores com um
alto poder de processamento. Um processador de um computador 
popular pode chegar a casa de bilhões de cálculos por segundo.

Como visto acima, essas equações descrevem com ajuda da computação
uma gama muito grande de problemas do mundo real; visando a importância
desse tipo de equação, em meados do século XVIII, ela se transformou em
uma disciplina independente. Ainda segundo \citeonline[p.~231]{sperandio2003calculo} ``no final desse 
mesmo século, a teoria das equações diferencias se transformou num
dos estudos mais importantes, em que se destacam as contribuições de 
Euler, Lagrange e Laplace''.

Vale ressaltar que existem equação diferenciais ordinárias (EDO) e
equações diferencias parciais (EDP) que usam mais de uma variável
independente. Para a resolução desta ultima equação são utilizados
métodos que não fazem parte do escopo deste trabalho.
 
Porém, é importante lembrar que esses métodos
como outros do cálculo numérico acumulam erros a cada iteração.
Uma boa implementação dos métodos é a chave para bons resultados
e eficiência. Cada linguagem de programação possui sua particularidade
e elas devem ser respeitadas nas implementações de cada método. 

Por tais razões apresentadas acimas, entre outras, o uso do cálculo
numérico para determinar a solução de equações diferenciais 
é de suma importância.
